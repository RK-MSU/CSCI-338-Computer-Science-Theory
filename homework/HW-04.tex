\documentclass[11pt]{article}

%Don't change any thing before \begin{document}
%In fact if you use sth fancy, you might need
%to add more packages, or macros.
\usepackage{amssymb,amsmath,amsthm}
\usepackage{times,psfrag,epsf,epsfig,graphics,graphicx}
\usepackage{algorithm}
\usepackage{algorithmic}

\setlength{\parskip}{1em}

\title{CSCI 338: Assignment~4}
\author{River Kelly}
\date{Thursday, April 1}

\begin{document}
\maketitle

\newpage
\section*{Problem 1}

Let ${\cal B}$ be the set of all infinite sequences over $\{a,b\}$. Show that
${\cal B}$ is uncountable, using a proof by diagonalization.

\begin{proof}
To prove that ${\cal B}$ is uncountable using diagonalization, we must show, by contradiction, that no correspondence exists between $\cal N$ and $\cal B$.
Suppose that a correspondence $f$ existed between $\cal N$ and $\cal B$.
For such correspondence to exist, $f$ must pair all the members of $\cal N$ with all the members of $\cal B$.
We must show that $f$ fails to work as it should by finding an $x$ in $\cal B$ that is not paired with anything in $\cal N$, a contradiction.

\noindent
Assuming that ${\cal B}$ is countable, the elements of ${\cal B}$ may be ordered as $b_{1}, b_{2}, b_{3}, \ldots b_{n}$, and a correspondence $f$ exists between $\cal N$ and $\cal B$, the following table shows a few values of a hypothetical correspondence.

%\Large
\begin{center}
\begin{tabular}{ c | c }
 $n$ & $f(n)$ \\
 \hline 
 1 & \underline{a}bbab$\ldots$ \\
 2 & b\underline{b}aab$\ldots$ \\
 3 & ab\underline{a}ba$\ldots$ \\
 4 & aaa\underline{b}a$\ldots$ \\
 $\vdots$ & $\vdots$
\end{tabular}
\end{center}
%\normalsize

\noindent
We can now construct the desired $x$ by taking the elements of the diagonal and complementing them so that to ensure that $x \neq f(n)$ for any $n$.
Continuing in this way down the diagonal of the table for $f$, we obtain all the digits of $x$.
Note that underlined values in the table above represent the construction for the inputs of $x$. Furthermore, the value of $x$ would be 
%\Large
$$x = \text{baba}\ldots$$
%\normalsize
But $x \in \cal B$, so then some $b_{i} = x$. But by the construction of $x$ is different from $b_{i}$ at the $i^{th}$ spot. A contradiction.

\noindent
$\therefore$ $\cal B$ is not countable. 
\end{proof}

\newpage
\section*{Problem 2}

Let $T=\{(i,j,k)|i,j,k\in{\cal N}\}$. Show that $T$ is countable.

\begin{proof}
To prove that $T$ is countable, we must show that there exists some correspondence $f$ between $\cal N$ and $T$. To do this, we will construct a list which shows a few values of a hypothetical correspondence $f$ between $\cal N$ and $T$.

\begin{center}
\begin{tabular}{ c | c }
 $n$ & $f(n)$ \\
 \hline 
 1 & $(0,0,0)$ \\
 2 & $(1,0,0)$ \\
 3 & $(0,1,0)$ \\
 4 & $(0,0,1)$ \\
 5 & $(2,0,0)$ \\
 6 & $(0,2,0)$ \\
 $\vdots$ & $\vdots$
\end{tabular}
\end{center}

\noindent
For each tuple $(i,j,k)$, let $s$ be the sum such that $s = i+j+k$ and $s \in \cal N$. This implies that we can enumerate all tuples in $T$ because there are a finite number of tuples whose sum is equal to $s$. This shows the existence of a one-to-one correspondence between $\cal N$ and $\cal T$.

\noindent
$\therefore$ $T$ must be countable.
\end{proof}

\newpage
\section*{Problem 3}

Let $INFINITE_{PDA}=\{<M>|M$ is a PDA and $L(M)$ is an infinite language$\}$.
Show that $INFINITE_{PDA}$ is decidable.

\begin{proof}
To prove that $INFINITE_{PDA}$ is decidable, we will construct a Turning Machine that decides it. Let $M$ denote such a Turning Machine, we can the convert $M$ into a Context Free Grammar (CFG) - lets call $N$.
The CFG $N$ can be converted into Chomsky Normal Form, called $N'$, and we can check if the there exists a derivation $A \Rightarrow uAv$ where $u,v \in \Sigma$.

\noindent
There is an infinite language $L(M)$ if $A \Rightarrow uAv$ is a derivation in $N'$, and there is not an infinite language $L(M)$ is there is not a derivation $A \Rightarrow uAv$ in $N'$. The Turning Machine $M$, which decides $INFINITE_{PDA}$ as follows:

\noindent
$M = $ "On input $<M>$ a PDA:
\newline
\noindent
1. Convert $M$ into an equivalent CFG $N$.
\newline
\noindent
2. Convert $N$ into an equivalent CFG $N'$ in Chomsky Normal Form.
\newline
\noindent
3. If the derivation $A \Rightarrow uAv$ is included in $N'$, accept, else, reject."

\noindent
$\therefore$  $INFINITE_{PDA}$ is decidable.
\end{proof}

% /* ------------ the above contents have been covered by March 12 ------------ */

\newpage
\section*{Problem 4}

Let $\Sigma=\{a,b\}$. Define the following language {\em ODD}$_{TM}$:

{\em ODD}$_{TM}=\{ <M>|M$ is a TM and $L(M)$ contains only strings of odd length $\}$.

\noindent
Prove that {\em ODD}$_{TM}$ is undecidable.

\begin{proof}
To prove {\em ODD}$_{TM}$ is undecidable, we will show that $A_{TM}$ reduces to {\em ODD}$_{TM}$. Assume that {\em ODD}$_{TM}$ is decidable with Turning Machine $R$. Let $< M , w >$ be the input into $A_{TM}$.

\noindent
First, we will construct Turning Machine $S$ on input $< M , w >$:

\noindent
$S = $ "On input $x$:\newline
1. If $|x|$ is odd, accept. \newline
2. If $|x|$ is even, run $M$ on $w$. If $M$ accepts, accept. Otherwise, reject."

\noindent
Now we can construct a TM to run $R$ on $< M , w >$ which decides $A_{TM}$;

\noindent
$H = $ "On input $< M , w >$:\newline
1. Run $R$ on $< M , w >$.\newline
2. If $R$ accepts, accept. If $R$ rejects, reject."

\noindent But $A_{TM}$ is undecidable, a contradiction.

\noindent $\therefore$ {\em ODD}$_{TM}$ is undecidable.
\end{proof}


\newpage
\section*{Problem 5}

Show that $EQ_{CFG}$ is undecidable.
\begin{proof}
To prove that $EQ_{CFG}$ is undecidable, we will construct a TM $S$ to decide $ALL_{CFG}$.

\noindent
Recall that,
$$ EQ_{CFG} = \{ \left< G_{1}, G_{2} \right> \mid G_{1} \text{ and } G_{2} \text{ are CFGs and } L(G_{1}) = L(G_{2})\} $$
and the $ALL_{CFG}$ will reduce to  $EQ_{CGF}$ where, 
$$ ALL_{CGF} = \{ \left< G \right> \mid G \text{ is a CFG and } L(G) = \Sigma^{*} \}$$
Let TM $R$ decide $EQ_{CFG}$ and the details of TM $S$ be:

\noindent
$S = $ "On input $G$ a CFG:\newline
1. Let $G_{0}$ be the CFG such that $L(G_{0}) = \Sigma^{*}$\newline
2. Run $R$ on input $\left< G , G_{0} \right>$.\newline
3. If $R$ accepts, accept. Otherwise reject."

\noindent
$ALL_{CFG}$ is undecidable, and TM $S$ decides $ALL_{CFG}$, a contradiction.

\noindent
$\therefore$ $EQ_{CGF}$ must also be undecidable.
\end{proof}

\newpage
\section*{Problem 6}

Show that $EQ_{CFG}$ is co-Turing-recognizable.

\begin{proof}
To prove that $EQ_{CFG}$ is co-Turing-recognizable we will construct a TM $S$ that recognizes $\overline{EQ_{CFG}}$.

\noindent
Recall that a language is co-Turing-recognizable if, and only if, its complement is a Turing-recognizable language.
$$ \overline{EQ_{CFG}} = \{ \left< G_{1}, G_{2} \right> \mid G_{1} \text{ and } G_{2} \text{ are CFGs and } L(G_{1}) \neq L(G_{2})\}$$

\noindent
The details of TM $S$ that recognizes $\overline{EQ_{CFG}}$:

\noindent
$S = $ "On input $\left< G_{1}, G_{2} \right>$, where $G_{1}$ and $G_{2}$ are CFGs:\newline
1. If at least one CFG, either $G_{1}$ and $G_{2}$, is invalid, accept.\newline
2. Convert the equivalent Chomsky Normal Form CFGs of $G_{1}$ and $G_{2}$, $G'_{1}$ and $G'_{2}$ respectively.\newline
3. Repeat step $4$ for $i = 1,2,3\ldots$\newline
4. Test $G'_{1}$ and $G'_{2}$ to generate a unique string $s \in \Sigma^{*}$\newline
If exactly one is valid and one is invalid, accept."

\noindent
$\therefore$ TM $S$ recognizes $\overline{EQ_{CFG}}$, thus $EQ_{CFG}$ is co-Turing-recognizable.
\end{proof}

\newpage
\section*{Problem 7}

Find a match in the following instance of the Post Correspondence Problem.
$$  \left\{
	\left[ \dfrac{ab}{abab} \right], 
	\left[ \dfrac{b}{a} \right], 
	\left[ \dfrac{aba}{b} \right],
	\left[ \dfrac{aa}{a} \right] \right\} $$

\begin{proof}
A matching result can be found/made/created by a combination of all the upper strings and all of the lower strings such that reading both of the top and bottom are the same. 

\noindent
Consider the collection as a list of dominos numbered 1-4. There is a match given the sequence 4,4,2 and 1. The match is show as:
$$ \left[ \dfrac{aa}{a} \right] , \left[ \dfrac{aa}{a} \right] , \left[ \dfrac{b}{a} \right] , \left[ \dfrac{ab}{abab} \right] = \dfrac{aaaabab}{aaaabab}$$
\end{proof}

\end{document}
