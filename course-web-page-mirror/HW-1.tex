\documentclass[11pt]{article}

%Don't change any thing before \begin{document}
%In fact if you use sth fancy, you might need
%to add more packages, or macros.
\usepackage{amssymb,amsmath}
\usepackage{times,psfrag,epsf,epsfig,graphics,graphicx}
\usepackage{algorithm}
\usepackage{algorithmic}


\begin{document}
\date{}
%\date{Feb 3, 2021}

\title{CSCI 338: Assignment~1~(7 points)}

%\author{Your Name Here}


\maketitle

%When writing up your solution, comment out the following until you reach Problem 1.
\noindent
This assignment is due on {\bf Thursday, Jan 28, 8:00pm}. You will need to
use Latex to generate a single pdf file and upload it under {\em Assignment 1}
on D2L. There will be a penalty for not using Latex (to finish the assignment).
This is {\bf not} a group-assignment, so you must finish the assignment by
yourself.

\section*{Problem 1}

Prove that $1^4+2^4+3^4+\cdots+n^4=\frac{n(n+1)(2n+1)(3n^2+3n-1)}{30}$.
\newline

%\noindent
%{\bf Proof:} ....
%....
%$\hfill \Box$
%\newline

\section*{Problem 2}

Given a planar graph $P=(V,E)$, we have Euler's formula:
$|V|+|F|-|E|=2$, where $F$ is the set of faces of $P$ and $E$ is the
set of edges of $P$.
Let $|V|=n$, where $V$ is the set of vertices of $P$.
Prove that $|F|$ is at most $2n$.
\newline

%\noindent
%{\bf Proof:} ....
%....
%$\hfill \Box$
%\newline

\section*{Problem 3}

Prove that in any simple graph there is a path from any vertex of odd degree
to some other vertex of odd degree.
\newline

%\noindent
%{\bf Proof:} ....
%....
%$\hfill \Box$
%\newline

\section*{Problem 4}

A fully binary tree $T$ is a tree such that all internal nodes have
two children. Prove that a fully binary tree with $n$ internal nodes
in total has $n+1$ leaves.
\newline

%\noindent
%{\bf Proof:} ....
%....
%draw/include a figure if necessary.
%$\hfill \Box$
%\newline

\section*{Problem 5}

Given an undirected graph $G=(V,E)$, the breadth-first-search starting at $v\in V$
($bfs(v)$ for short) is to generate a shortest path tree starting at vertex
$v\in V$. The diameter of $G$ is the longest of all shortest paths $\delta(u,v), u,v\in V$.
\newline

When $G$ is a tree, the following algorithm is proposed to compute the
diameter of $G$.
\newline

1. Run $bfs(w), w\in V$, and compute the vertex $x\in V$ furthest from $w$.

2. Run $bfs(x)$ and compute the vertex $y\in V$ furthest from $x$.

3. Return $\delta(x,y)$ as the diameter of $G$.
\newline

Prove that this algorithm is correct; i.e., $\delta(x,y)$ is in fact the
longest among all the shortest paths between $u,v\in V$.
\newline

%\noindent
%{\bf Proof:} ....
%....
%draw/include a figure if necessary.
%$\hfill \Box$

\end{document}

