\documentclass[11pt]{article}

\usepackage{amssymb,amsmath,amsthm}
\usepackage{times,psfrag,epsf,epsfig,graphics,graphicx}
\usepackage{algorithm}
\usepackage{algorithmic}

\usepackage{caption}
\usepackage{subcaption}

\usepackage{tikz}
%\usepackage{pgfplots}
%\pgfplotsset{compat=1.17}

\title{CSCI 338: Quiz 3}
\author{River Kelly}
\date{Friday, March 19}

\begin{document}
\maketitle
\section*{Problem 1}
Based on the reduction from Sorting to 2D Convex Hull that was covered on March
15, suppose that the input for Sorting are given as 
$x_{1} = 4$, 
$x_{2} = -3$, 
$x_{3} = 1$, 
$x_{4} = 0$, 
$x_{5} = 3$, 
$x_{6} = -4$, 
$x_{7} = -2$.
\newline


\vspace{5pt}
\noindent
\textbf{(1)} List the points constructed for the 2D Convex Hull problem. (You must list the
coordinates of the points.)
\newline
\noindent
\begin{figure}[h]
\centering
\begin{subfigure}{.5\textwidth}
    \centering
    \small
    \begin{center}
    \begin{tabular}{c|c|c}
     $x_{n}$ & Value & Point Coordinate\\
     \hline 
     $x_{1}$ & 4 & $(1,4)$ \\
     $x_{2}$ & -3 & $(2,-3)$ \\
     $x_{3}$ & 1 & $(3,1)$ \\
     $x_{4}$ & 0 & $(4,0)$ \\
     $x_{5}$ & 3 & $(5,3)$ \\
     $x_{6}$ & -4 & $(6,-4)$ \\
     $x_{7}$ & -2 & $(7,-2)$ \\
    \end{tabular}
    \end{center}
    \normalsize
\end{subfigure}%
\begin{subfigure}{.5\textwidth}
    \centering
    \begin{tikzpicture}[x=1cm,y=0.4cm]
    \draw[latex-latex, thin, draw=gray] (-0.5,0)--(7.5,0) node [right] {$x$};
    \draw[latex-latex, thin, draw=gray] (0,-4.5)--(0,4.5) node [above] {$y$};
    
    \node [label={170:{\scriptsize(1,4)}}, red] at (1,3.9) {\textbullet};
    \node [label={170:{\scriptsize(2,-3)}}, red] at (2,-3.1) {\textbullet};
    \node [label={170:{\scriptsize(3,1)}}, black] at (3,0.9) {\textbullet};
    \node [label={70:{\scriptsize(4,0)}}, black] at (4,-0.1) {\textbullet};
    \node [label={170:{\scriptsize(5,3)}}, red] at (5,2.9) {\textbullet};
    \node [label={170:{\scriptsize(6,-4)}}, red] at (6,-4.1) {\textbullet};
    \node [label={170:{\scriptsize(7,-2)}}, red] at (7,-2.1) {\textbullet};
    
    \draw [dotted, gray] (0, -4)--(7,-4);
    \draw [dotted, gray] (0, -3)--(7,-3);
    \draw [dotted, gray] (0, -2)--(7,-2);
    \draw [dotted, gray] (0, -1)--(7,-1);
    \draw [dotted, gray] (0, 1)--(7,1);
    \draw [dotted, gray] (0, 2)--(7,2);
    \draw [dotted, gray] (0, 3)--(7,3);
    \draw [dotted, gray] (0, 4)--(7,4);
    
    \draw [dotted, gray] (1, -4)--(1,4);
    \draw [dotted, gray] (2, -4)--(2,4);
    \draw [dotted, gray] (3, -4)--(3,4);
    \draw [dotted, gray] (4, -4)--(4,4);
    \draw [dotted, gray] (5, -4)--(5,4);
    \draw [dotted, gray] (6, -4)--(6,4);
    \draw [dotted, gray] (7, -4)--(7,4);
    \end{tikzpicture}
\end{subfigure}
\end{figure}
\newline
\noindent
\textit{Note:} The red points represent the points used to create the perimeter of the 2D convex hull.
\newline

\vspace{5pt}
\noindent
\textbf{(2)} Briefly show how the sorted points $x_{i}$’s are obtained once the 2D convex hull
is given.
\newline

\vspace{5pt}
\noindent
Once the convex hull is given, the sorted points $x_{i}$'s are obtained by; starting with the leftmost vertex, the hull then lists the vertices in counterclockwise order (i.e. sorted order).

\end{document}
