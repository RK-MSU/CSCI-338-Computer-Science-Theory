\documentclass[11pt]{article}

\usepackage{amssymb,amsmath}
\usepackage{amsthm}
\usepackage{times,psfrag,epsf,epsfig,graphics,graphicx}
\usepackage{algorithm}
\usepackage{algorithmic}

\title{CSCI 338: Quiz 1}
\author{River Kelly}
\date{Friday, March 5}

\begin{document}
\maketitle

\newpage
\section*{Problem 1}
Given the set $A=\{-36,-25,-16,-9,-4,1,4,9,16,25,36\}$, is $A$ countable? Why?
\newline
\newline
Set $A$ is countable.
\newline
\newline
By definition, set $A$ is countable if either it is finite or it has the same size as $N$ (the set of natural numbers).
\newline
\newline
Set $A$ is a finite set, therefore it is countable.

\newpage
\section*{Problem 2}
Let $B$ be the set of all complete graphs. Is $B$ countable? Why?
\newline
\newline
Set $B$ is countable.
\newline
\newline
By definition, a complete graph is a simple undirected graph in which every pair of distinct vertices is connected by a unique edge.
\newline
\newline
By definition, a set is countable if either it is finite or it has the same size as $N$ (the set of natural numbers, i.e 1 to 1 mapping correspondence).
\newline
\newline
If set $B$ is the set of all complete graphs, then set $B$ is NOT finite, it is an infinite set. Then, for set $B$ to be infinitely countable, it must have a 1-1 correspondence with $N$ (the set of natural numbers).
\newline
\newline
If $a_1$ is the first element in set $B$, and $a_2$ is the 2nd element in $B$, such that $a_{2}$ has 1 more vertex than $a_{1}$.... and so on .... $B = \{a_{1},a_{2},...,{a_n}\}$. Then we can see that set $B$ can be mapped in a 1-1 correspondence with $N$ (the set of natural numbers).
\newline
\newline
Therefore, set $B$ is infinitely countable.




\end{document}
