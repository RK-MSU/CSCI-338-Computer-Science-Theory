\documentclass[11pt]{article}

\usepackage{amssymb,amsmath,amsthm}
\usepackage{times,psfrag,epsf,epsfig,graphics,graphicx}
\usepackage{algorithm}
\usepackage{algorithmic}

\setlength{\parskip}{1em}

\title{CSCI 338: Quiz~7}
\author{River Kelly}
\date{Friday, April 16}

\begin{document}
\maketitle

\newpage
\section*{Problem 1}

On Apr 09 and 12, we covered some problems in NP. On Apr 14-16, we covered
some NP-complete problems.

\noindent
In your own language, what is the difference between NP and NP-complete?

\subsection*{Response}

NP and NP-complete are very closely related. In fact, all NP-complete problems fall in the set of the NP class. This means that the solution to NP-complete problems can be verified in NP, but it is at least as difficult NP-hard. One of the identifying characteristics of NP-complete is such that individual complexity is related to that of the entire class. If an algorithm has polynomial time complexity for such a problem in NP, then NP would be solvable in P time (i.e. NP-complete).

\noindent
An NP-complete problem is a problem such that it may be reduced to another problem in NP. This is how/why NP-complete problems are shared in both NP and NP-hard.

\end{document}

